\documentclass[12pt]{article}
\usepackage{graphicx} % Required for inserting images
\usepackage{amsmath}
\usepackage{amssymb}
%\usepackage[italian]{babel}
\usepackage{hyperref}%
\usepackage{mathrsfs}

\begin{document}

\section*{}
Marco Reina 7066486 \newline
Samuele Serri 7069839 \newline
\section*{}
a)
for class y=0
E[x1] = (1+1+2+3+3)/5 = 2
E[x2] = (1+1+2+2+3)/5 = 1.8
cov(x1,x2) = E[x1x2] - E[x1]E[x2] = (3+6+6+1+2)/5 - 3.6 = 0
for class y=1
E[x1] = (1+2+4+5+5)/5 = 3.4
E[x2] = (4+5+6+6+7)/5 = 5.6
cov(x1,x2) = E[x1x2] - E[x1]E[x2] = (30+24+20+10+7)/5 - 19.04 = 18.2-19.04 = -0.84
b)
xT = (3.5,2)
sigma = covariance matrix
for y=0
var(x1) = E[x1^2] - E[x1]^2 = (1+1+4+9+9)/5 - 4 = 0.8
var(x2) = (1+1+4+4+9)/5 - 3.6 = 0.2
for y=1
var(x1) = (1+4+16+25+25)/5 - 11.56 = 14.2 - 11.56 = 2.64
var(x2) = (16+25+36+36+49)/5 - 31.36 = 32.4 - 31.36 = 1.04
sigma for y=0
(0.8, 0)
(0, 0.2)
sigma for y=1
(2.64, -0.84)
(-0.84, 1.04)

mu = vector of expected values

c)
LDA assumes that the classes have different means and shared variance,
while with QDA each class can have a different variance

d)
???_

e)
LDA is a much less flexible classifier than QDA.
therefore LDA usually makes better predictions when there are
relatively few training observations and reducing variance is crucial.
QDA can be used with a bigger sample size, when the variance of the
classifier is not a huge concern.

numver of features??
\end{document}