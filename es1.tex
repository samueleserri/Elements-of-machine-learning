\documentclass[12pt]{article}
\usepackage{graphicx} % Required for inserting images
\usepackage{amsmath}
\usepackage{amssymb}
%\usepackage[italian]{babel}
\usepackage{hyperref}%
\usepackage{mathrsfs}

\begin{document}

\section*{}
Marco Reina 7066486 \newline
Samuele Serri 7069839 \newline
\section*{}

Given a set of $n$ points $\{(x_1, y_1), ..., (x_n, y_n)\}$,
parametric and non parametric methods are used to find a function $f$ to fit the given data,
such that for any pair $(X,Y)$, we get $Y = f(X)$. \newline

Parametric methods are simple but inflexible because the number of parameters is decided beforehand.
Nonparametric methods instead are more flexible at the cost of a higher complexity. \newline

Parametric methods assume the functional form and only the parameters are to be computed.
With nonparametric methods we try to find the true function directly, having to learn the form as well as the parameters. \newline

With small datasets parametric methods are preferred
because their simplicity makes them less sensitive to overfitting,
while nonparametric methods are more suitable for big datasets.

\end{document}
