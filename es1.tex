\documentclass[12pt]{article}
\usepackage{graphicx} % Required for inserting images
\usepackage{amsmath}
\usepackage{amssymb}
\usepackage{hyperref}%
\usepackage{mathrsfs}



\begin{document}


\section*{}
Marco Reina 7066486 \newline
Samuele Serri 7069839 \newline
\section*{a)}
In logistic regression model the function:
\begin{equation*}
    Y = \frac{e^{f(x)}}{1 + e^{f(x)}}
\end{equation*}
it is used for mapping the linear predictor $f(x)$ into $[0, 1]$ so the output can be interpreted as a probability.
\section{b)}
\begin{equation*}
    p(y_1, ..., y_n| \beta_0, \beta_1, ...,\beta_n, x_1, ..., x_n) = \prod_{i : y_i= 1}^n \frac{e^{f(x_i)}}{1 + e^{f(x_i)}}\prod_{i : y_i = 0}^n 1 - \frac{e^{f(x_i)}}{1 + e^{f(x_i)}}
\end{equation*}
If we take the logarithm: \begin{equation*}
    \sum_{i: y_i = 1}^n{log\bigg(\frac{e^{f(x_i)}}{1 + e^{f(x_i)}}\bigg)} + \sum_{i:y_i = 0}^nlog\bigg( 1 - \frac{e^{f(x_i)}}{1 + e^{f(x_i)}}\bigg) = 
\end{equation*}
\begin{equation*}
    =  \sum_{i: y_i = 1}^n f(x_i) -  \sum_{i: y_i = 1}^nlog(1 + e^{f(x_i)}) - \sum_{i:y_i = 0}^nlog(1 + e^{f(x_i)})
\end{equation*}
In order to estimate the parameters of the model we need to find the maximum of the likelihood function. Finding the maximum of the log-likelihood function is equivalent because the logarithmic transformation is monotonic and preserves maximums.
\section*{c)}
For classification problems discriminative models return a function that defines strict boundaries between the classes. Whereas generative models return the probability $p_g(x)$ of $x$ to belonging in the class $g$ for each class $g \in G$ . \newline
Generative models estimates their parameters maximizing a likelihood function while discriminative models minimize a loss function. 
\end{document}
